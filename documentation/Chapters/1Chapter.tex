The system for both the data acquisition and data processing is composed of a Lego Mindstorm and linux o.s. pc. The code language for the brick to be programmed is the c, which is a well known language and permits to the developer a simple debugging process. \\ Once the program has been compiled and uploaded on the brick (instruction on nxtOSEK website), some steps are needed for the data to be transferred. \\
\begin{itemize}
\item launch the BROFist file with the command '-l', this in order to get the bluetooth identification key of the brick;
\item launch the BROFist file with the command '-m' followed by the bluetooth key gets at the previous step, this let the bluetooth data transfer from the brick to the pc (and viceversa) possible;   
\item select the 'BROclient' program and run it by pressing the run button;
\end{itemize}

While the program is running, on the brick monitor is shown information about the software status:
\begin{itemize}
\item name and release of the program; 
\item running phase: every 5 second the motor velocity changes (interval from -100 to 100), this permits to get data from a vast range of velocity;  
\item motor status: information about the motor movement, whether it is running or is not; 
\end{itemize}
The program stops at any time if the ENTER button (that orange) is pressed; this stop the program by invoking the TaskTerminate() method. 